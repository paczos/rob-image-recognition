
% Default to the notebook output style

    


% Inherit from the specified cell style.




    
\documentclass[11pt]{article}

    
    
    \usepackage[T1]{fontenc}
    % Nicer default font (+ math font) than Computer Modern for most use cases
    \usepackage{mathpazo}

    % Basic figure setup, for now with no caption control since it's done
    % automatically by Pandoc (which extracts ![](path) syntax from Markdown).
    \usepackage{graphicx}
    % We will generate all images so they have a width \maxwidth. This means
    % that they will get their normal width if they fit onto the page, but
    % are scaled down if they would overflow the margins.
    \makeatletter
    \def\maxwidth{\ifdim\Gin@nat@width>\linewidth\linewidth
    \else\Gin@nat@width\fi}
    \makeatother
    \let\Oldincludegraphics\includegraphics
    % Set max figure width to be 80% of text width, for now hardcoded.
    \renewcommand{\includegraphics}[1]{\Oldincludegraphics[width=.8\maxwidth]{#1}}
    % Ensure that by default, figures have no caption (until we provide a
    % proper Figure object with a Caption API and a way to capture that
    % in the conversion process - todo).
    \usepackage{caption}
    \DeclareCaptionLabelFormat{nolabel}{}
    \captionsetup{labelformat=nolabel}

    \usepackage{adjustbox} % Used to constrain images to a maximum size 
    \usepackage{xcolor} % Allow colors to be defined
    \usepackage{enumerate} % Needed for markdown enumerations to work
    \usepackage{geometry} % Used to adjust the document margins
    \usepackage{amsmath} % Equations
    \usepackage{amssymb} % Equations
    \usepackage{textcomp} % defines textquotesingle
    % Hack from http://tex.stackexchange.com/a/47451/13684:
    \AtBeginDocument{%
        \def\PYZsq{\textquotesingle}% Upright quotes in Pygmentized code
    }
    \usepackage{upquote} % Upright quotes for verbatim code
    \usepackage{eurosym} % defines \euro
    \usepackage[mathletters]{ucs} % Extended unicode (utf-8) support
    \usepackage[utf8x]{inputenc} % Allow utf-8 characters in the tex document
    \usepackage{fancyvrb} % verbatim replacement that allows latex
    \usepackage{grffile} % extends the file name processing of package graphics 
                         % to support a larger range 
    % The hyperref package gives us a pdf with properly built
    % internal navigation ('pdf bookmarks' for the table of contents,
    % internal cross-reference links, web links for URLs, etc.)
    \usepackage{hyperref}
    \usepackage{longtable} % longtable support required by pandoc >1.10
    \usepackage{booktabs}  % table support for pandoc > 1.12.2
    \usepackage[inline]{enumitem} % IRkernel/repr support (it uses the enumerate* environment)
    \usepackage[normalem]{ulem} % ulem is needed to support strikethroughs (\sout)
                                % normalem makes italics be italics, not underlines
    \usepackage{mathrsfs}
    

    
    
    % Colors for the hyperref package
    \definecolor{urlcolor}{rgb}{0,.145,.698}
    \definecolor{linkcolor}{rgb}{.71,0.21,0.01}
    \definecolor{citecolor}{rgb}{.12,.54,.11}

    % ANSI colors
    \definecolor{ansi-black}{HTML}{3E424D}
    \definecolor{ansi-black-intense}{HTML}{282C36}
    \definecolor{ansi-red}{HTML}{E75C58}
    \definecolor{ansi-red-intense}{HTML}{B22B31}
    \definecolor{ansi-green}{HTML}{00A250}
    \definecolor{ansi-green-intense}{HTML}{007427}
    \definecolor{ansi-yellow}{HTML}{DDB62B}
    \definecolor{ansi-yellow-intense}{HTML}{B27D12}
    \definecolor{ansi-blue}{HTML}{208FFB}
    \definecolor{ansi-blue-intense}{HTML}{0065CA}
    \definecolor{ansi-magenta}{HTML}{D160C4}
    \definecolor{ansi-magenta-intense}{HTML}{A03196}
    \definecolor{ansi-cyan}{HTML}{60C6C8}
    \definecolor{ansi-cyan-intense}{HTML}{258F8F}
    \definecolor{ansi-white}{HTML}{C5C1B4}
    \definecolor{ansi-white-intense}{HTML}{A1A6B2}
    \definecolor{ansi-default-inverse-fg}{HTML}{FFFFFF}
    \definecolor{ansi-default-inverse-bg}{HTML}{000000}

    % commands and environments needed by pandoc snippets
    % extracted from the output of `pandoc -s`
    \providecommand{\tightlist}{%
      \setlength{\itemsep}{0pt}\setlength{\parskip}{0pt}}
    \DefineVerbatimEnvironment{Highlighting}{Verbatim}{commandchars=\\\{\}}
    % Add ',fontsize=\small' for more characters per line
    \newenvironment{Shaded}{}{}
    \newcommand{\KeywordTok}[1]{\textcolor[rgb]{0.00,0.44,0.13}{\textbf{{#1}}}}
    \newcommand{\DataTypeTok}[1]{\textcolor[rgb]{0.56,0.13,0.00}{{#1}}}
    \newcommand{\DecValTok}[1]{\textcolor[rgb]{0.25,0.63,0.44}{{#1}}}
    \newcommand{\BaseNTok}[1]{\textcolor[rgb]{0.25,0.63,0.44}{{#1}}}
    \newcommand{\FloatTok}[1]{\textcolor[rgb]{0.25,0.63,0.44}{{#1}}}
    \newcommand{\CharTok}[1]{\textcolor[rgb]{0.25,0.44,0.63}{{#1}}}
    \newcommand{\StringTok}[1]{\textcolor[rgb]{0.25,0.44,0.63}{{#1}}}
    \newcommand{\CommentTok}[1]{\textcolor[rgb]{0.38,0.63,0.69}{\textit{{#1}}}}
    \newcommand{\OtherTok}[1]{\textcolor[rgb]{0.00,0.44,0.13}{{#1}}}
    \newcommand{\AlertTok}[1]{\textcolor[rgb]{1.00,0.00,0.00}{\textbf{{#1}}}}
    \newcommand{\FunctionTok}[1]{\textcolor[rgb]{0.02,0.16,0.49}{{#1}}}
    \newcommand{\RegionMarkerTok}[1]{{#1}}
    \newcommand{\ErrorTok}[1]{\textcolor[rgb]{1.00,0.00,0.00}{\textbf{{#1}}}}
    \newcommand{\NormalTok}[1]{{#1}}
    
    % Additional commands for more recent versions of Pandoc
    \newcommand{\ConstantTok}[1]{\textcolor[rgb]{0.53,0.00,0.00}{{#1}}}
    \newcommand{\SpecialCharTok}[1]{\textcolor[rgb]{0.25,0.44,0.63}{{#1}}}
    \newcommand{\VerbatimStringTok}[1]{\textcolor[rgb]{0.25,0.44,0.63}{{#1}}}
    \newcommand{\SpecialStringTok}[1]{\textcolor[rgb]{0.73,0.40,0.53}{{#1}}}
    \newcommand{\ImportTok}[1]{{#1}}
    \newcommand{\DocumentationTok}[1]{\textcolor[rgb]{0.73,0.13,0.13}{\textit{{#1}}}}
    \newcommand{\AnnotationTok}[1]{\textcolor[rgb]{0.38,0.63,0.69}{\textbf{\textit{{#1}}}}}
    \newcommand{\CommentVarTok}[1]{\textcolor[rgb]{0.38,0.63,0.69}{\textbf{\textit{{#1}}}}}
    \newcommand{\VariableTok}[1]{\textcolor[rgb]{0.10,0.09,0.49}{{#1}}}
    \newcommand{\ControlFlowTok}[1]{\textcolor[rgb]{0.00,0.44,0.13}{\textbf{{#1}}}}
    \newcommand{\OperatorTok}[1]{\textcolor[rgb]{0.40,0.40,0.40}{{#1}}}
    \newcommand{\BuiltInTok}[1]{{#1}}
    \newcommand{\ExtensionTok}[1]{{#1}}
    \newcommand{\PreprocessorTok}[1]{\textcolor[rgb]{0.74,0.48,0.00}{{#1}}}
    \newcommand{\AttributeTok}[1]{\textcolor[rgb]{0.49,0.56,0.16}{{#1}}}
    \newcommand{\InformationTok}[1]{\textcolor[rgb]{0.38,0.63,0.69}{\textbf{\textit{{#1}}}}}
    \newcommand{\WarningTok}[1]{\textcolor[rgb]{0.38,0.63,0.69}{\textbf{\textit{{#1}}}}}
    
    
    % Define a nice break command that doesn't care if a line doesn't already
    % exist.
    \def\br{\hspace*{\fill} \\* }
    % Math Jax compatibility definitions
    \def\gt{>}
    \def\lt{<}
    \let\Oldtex\TeX
    \let\Oldlatex\LaTeX
    \renewcommand{\TeX}{\textrm{\Oldtex}}
    \renewcommand{\LaTeX}{\textrm{\Oldlatex}}
    % Document parameters
    % Document title
    \title{ROB Klasyfikacja Bayesa. Paweł Paczuski (271082)}
    
    
    
    
    

    % Pygments definitions
    
\makeatletter
\def\PY@reset{\let\PY@it=\relax \let\PY@bf=\relax%
    \let\PY@ul=\relax \let\PY@tc=\relax%
    \let\PY@bc=\relax \let\PY@ff=\relax}
\def\PY@tok#1{\csname PY@tok@#1\endcsname}
\def\PY@toks#1+{\ifx\relax#1\empty\else%
    \PY@tok{#1}\expandafter\PY@toks\fi}
\def\PY@do#1{\PY@bc{\PY@tc{\PY@ul{%
    \PY@it{\PY@bf{\PY@ff{#1}}}}}}}
\def\PY#1#2{\PY@reset\PY@toks#1+\relax+\PY@do{#2}}

\expandafter\def\csname PY@tok@w\endcsname{\def\PY@tc##1{\textcolor[rgb]{0.73,0.73,0.73}{##1}}}
\expandafter\def\csname PY@tok@c\endcsname{\let\PY@it=\textit\def\PY@tc##1{\textcolor[rgb]{0.25,0.50,0.50}{##1}}}
\expandafter\def\csname PY@tok@cp\endcsname{\def\PY@tc##1{\textcolor[rgb]{0.74,0.48,0.00}{##1}}}
\expandafter\def\csname PY@tok@k\endcsname{\let\PY@bf=\textbf\def\PY@tc##1{\textcolor[rgb]{0.00,0.50,0.00}{##1}}}
\expandafter\def\csname PY@tok@kp\endcsname{\def\PY@tc##1{\textcolor[rgb]{0.00,0.50,0.00}{##1}}}
\expandafter\def\csname PY@tok@kt\endcsname{\def\PY@tc##1{\textcolor[rgb]{0.69,0.00,0.25}{##1}}}
\expandafter\def\csname PY@tok@o\endcsname{\def\PY@tc##1{\textcolor[rgb]{0.40,0.40,0.40}{##1}}}
\expandafter\def\csname PY@tok@ow\endcsname{\let\PY@bf=\textbf\def\PY@tc##1{\textcolor[rgb]{0.67,0.13,1.00}{##1}}}
\expandafter\def\csname PY@tok@nb\endcsname{\def\PY@tc##1{\textcolor[rgb]{0.00,0.50,0.00}{##1}}}
\expandafter\def\csname PY@tok@nf\endcsname{\def\PY@tc##1{\textcolor[rgb]{0.00,0.00,1.00}{##1}}}
\expandafter\def\csname PY@tok@nc\endcsname{\let\PY@bf=\textbf\def\PY@tc##1{\textcolor[rgb]{0.00,0.00,1.00}{##1}}}
\expandafter\def\csname PY@tok@nn\endcsname{\let\PY@bf=\textbf\def\PY@tc##1{\textcolor[rgb]{0.00,0.00,1.00}{##1}}}
\expandafter\def\csname PY@tok@ne\endcsname{\let\PY@bf=\textbf\def\PY@tc##1{\textcolor[rgb]{0.82,0.25,0.23}{##1}}}
\expandafter\def\csname PY@tok@nv\endcsname{\def\PY@tc##1{\textcolor[rgb]{0.10,0.09,0.49}{##1}}}
\expandafter\def\csname PY@tok@no\endcsname{\def\PY@tc##1{\textcolor[rgb]{0.53,0.00,0.00}{##1}}}
\expandafter\def\csname PY@tok@nl\endcsname{\def\PY@tc##1{\textcolor[rgb]{0.63,0.63,0.00}{##1}}}
\expandafter\def\csname PY@tok@ni\endcsname{\let\PY@bf=\textbf\def\PY@tc##1{\textcolor[rgb]{0.60,0.60,0.60}{##1}}}
\expandafter\def\csname PY@tok@na\endcsname{\def\PY@tc##1{\textcolor[rgb]{0.49,0.56,0.16}{##1}}}
\expandafter\def\csname PY@tok@nt\endcsname{\let\PY@bf=\textbf\def\PY@tc##1{\textcolor[rgb]{0.00,0.50,0.00}{##1}}}
\expandafter\def\csname PY@tok@nd\endcsname{\def\PY@tc##1{\textcolor[rgb]{0.67,0.13,1.00}{##1}}}
\expandafter\def\csname PY@tok@s\endcsname{\def\PY@tc##1{\textcolor[rgb]{0.73,0.13,0.13}{##1}}}
\expandafter\def\csname PY@tok@sd\endcsname{\let\PY@it=\textit\def\PY@tc##1{\textcolor[rgb]{0.73,0.13,0.13}{##1}}}
\expandafter\def\csname PY@tok@si\endcsname{\let\PY@bf=\textbf\def\PY@tc##1{\textcolor[rgb]{0.73,0.40,0.53}{##1}}}
\expandafter\def\csname PY@tok@se\endcsname{\let\PY@bf=\textbf\def\PY@tc##1{\textcolor[rgb]{0.73,0.40,0.13}{##1}}}
\expandafter\def\csname PY@tok@sr\endcsname{\def\PY@tc##1{\textcolor[rgb]{0.73,0.40,0.53}{##1}}}
\expandafter\def\csname PY@tok@ss\endcsname{\def\PY@tc##1{\textcolor[rgb]{0.10,0.09,0.49}{##1}}}
\expandafter\def\csname PY@tok@sx\endcsname{\def\PY@tc##1{\textcolor[rgb]{0.00,0.50,0.00}{##1}}}
\expandafter\def\csname PY@tok@m\endcsname{\def\PY@tc##1{\textcolor[rgb]{0.40,0.40,0.40}{##1}}}
\expandafter\def\csname PY@tok@gh\endcsname{\let\PY@bf=\textbf\def\PY@tc##1{\textcolor[rgb]{0.00,0.00,0.50}{##1}}}
\expandafter\def\csname PY@tok@gu\endcsname{\let\PY@bf=\textbf\def\PY@tc##1{\textcolor[rgb]{0.50,0.00,0.50}{##1}}}
\expandafter\def\csname PY@tok@gd\endcsname{\def\PY@tc##1{\textcolor[rgb]{0.63,0.00,0.00}{##1}}}
\expandafter\def\csname PY@tok@gi\endcsname{\def\PY@tc##1{\textcolor[rgb]{0.00,0.63,0.00}{##1}}}
\expandafter\def\csname PY@tok@gr\endcsname{\def\PY@tc##1{\textcolor[rgb]{1.00,0.00,0.00}{##1}}}
\expandafter\def\csname PY@tok@ge\endcsname{\let\PY@it=\textit}
\expandafter\def\csname PY@tok@gs\endcsname{\let\PY@bf=\textbf}
\expandafter\def\csname PY@tok@gp\endcsname{\let\PY@bf=\textbf\def\PY@tc##1{\textcolor[rgb]{0.00,0.00,0.50}{##1}}}
\expandafter\def\csname PY@tok@go\endcsname{\def\PY@tc##1{\textcolor[rgb]{0.53,0.53,0.53}{##1}}}
\expandafter\def\csname PY@tok@gt\endcsname{\def\PY@tc##1{\textcolor[rgb]{0.00,0.27,0.87}{##1}}}
\expandafter\def\csname PY@tok@err\endcsname{\def\PY@bc##1{\setlength{\fboxsep}{0pt}\fcolorbox[rgb]{1.00,0.00,0.00}{1,1,1}{\strut ##1}}}
\expandafter\def\csname PY@tok@kc\endcsname{\let\PY@bf=\textbf\def\PY@tc##1{\textcolor[rgb]{0.00,0.50,0.00}{##1}}}
\expandafter\def\csname PY@tok@kd\endcsname{\let\PY@bf=\textbf\def\PY@tc##1{\textcolor[rgb]{0.00,0.50,0.00}{##1}}}
\expandafter\def\csname PY@tok@kn\endcsname{\let\PY@bf=\textbf\def\PY@tc##1{\textcolor[rgb]{0.00,0.50,0.00}{##1}}}
\expandafter\def\csname PY@tok@kr\endcsname{\let\PY@bf=\textbf\def\PY@tc##1{\textcolor[rgb]{0.00,0.50,0.00}{##1}}}
\expandafter\def\csname PY@tok@bp\endcsname{\def\PY@tc##1{\textcolor[rgb]{0.00,0.50,0.00}{##1}}}
\expandafter\def\csname PY@tok@fm\endcsname{\def\PY@tc##1{\textcolor[rgb]{0.00,0.00,1.00}{##1}}}
\expandafter\def\csname PY@tok@vc\endcsname{\def\PY@tc##1{\textcolor[rgb]{0.10,0.09,0.49}{##1}}}
\expandafter\def\csname PY@tok@vg\endcsname{\def\PY@tc##1{\textcolor[rgb]{0.10,0.09,0.49}{##1}}}
\expandafter\def\csname PY@tok@vi\endcsname{\def\PY@tc##1{\textcolor[rgb]{0.10,0.09,0.49}{##1}}}
\expandafter\def\csname PY@tok@vm\endcsname{\def\PY@tc##1{\textcolor[rgb]{0.10,0.09,0.49}{##1}}}
\expandafter\def\csname PY@tok@sa\endcsname{\def\PY@tc##1{\textcolor[rgb]{0.73,0.13,0.13}{##1}}}
\expandafter\def\csname PY@tok@sb\endcsname{\def\PY@tc##1{\textcolor[rgb]{0.73,0.13,0.13}{##1}}}
\expandafter\def\csname PY@tok@sc\endcsname{\def\PY@tc##1{\textcolor[rgb]{0.73,0.13,0.13}{##1}}}
\expandafter\def\csname PY@tok@dl\endcsname{\def\PY@tc##1{\textcolor[rgb]{0.73,0.13,0.13}{##1}}}
\expandafter\def\csname PY@tok@s2\endcsname{\def\PY@tc##1{\textcolor[rgb]{0.73,0.13,0.13}{##1}}}
\expandafter\def\csname PY@tok@sh\endcsname{\def\PY@tc##1{\textcolor[rgb]{0.73,0.13,0.13}{##1}}}
\expandafter\def\csname PY@tok@s1\endcsname{\def\PY@tc##1{\textcolor[rgb]{0.73,0.13,0.13}{##1}}}
\expandafter\def\csname PY@tok@mb\endcsname{\def\PY@tc##1{\textcolor[rgb]{0.40,0.40,0.40}{##1}}}
\expandafter\def\csname PY@tok@mf\endcsname{\def\PY@tc##1{\textcolor[rgb]{0.40,0.40,0.40}{##1}}}
\expandafter\def\csname PY@tok@mh\endcsname{\def\PY@tc##1{\textcolor[rgb]{0.40,0.40,0.40}{##1}}}
\expandafter\def\csname PY@tok@mi\endcsname{\def\PY@tc##1{\textcolor[rgb]{0.40,0.40,0.40}{##1}}}
\expandafter\def\csname PY@tok@il\endcsname{\def\PY@tc##1{\textcolor[rgb]{0.40,0.40,0.40}{##1}}}
\expandafter\def\csname PY@tok@mo\endcsname{\def\PY@tc##1{\textcolor[rgb]{0.40,0.40,0.40}{##1}}}
\expandafter\def\csname PY@tok@ch\endcsname{\let\PY@it=\textit\def\PY@tc##1{\textcolor[rgb]{0.25,0.50,0.50}{##1}}}
\expandafter\def\csname PY@tok@cm\endcsname{\let\PY@it=\textit\def\PY@tc##1{\textcolor[rgb]{0.25,0.50,0.50}{##1}}}
\expandafter\def\csname PY@tok@cpf\endcsname{\let\PY@it=\textit\def\PY@tc##1{\textcolor[rgb]{0.25,0.50,0.50}{##1}}}
\expandafter\def\csname PY@tok@c1\endcsname{\let\PY@it=\textit\def\PY@tc##1{\textcolor[rgb]{0.25,0.50,0.50}{##1}}}
\expandafter\def\csname PY@tok@cs\endcsname{\let\PY@it=\textit\def\PY@tc##1{\textcolor[rgb]{0.25,0.50,0.50}{##1}}}

\def\PYZbs{\char`\\}
\def\PYZus{\char`\_}
\def\PYZob{\char`\{}
\def\PYZcb{\char`\}}
\def\PYZca{\char`\^}
\def\PYZam{\char`\&}
\def\PYZlt{\char`\<}
\def\PYZgt{\char`\>}
\def\PYZsh{\char`\#}
\def\PYZpc{\char`\%}
\def\PYZdl{\char`\$}
\def\PYZhy{\char`\-}
\def\PYZsq{\char`\'}
\def\PYZdq{\char`\"}
\def\PYZti{\char`\~}
% for compatibility with earlier versions
\def\PYZat{@}
\def\PYZlb{[}
\def\PYZrb{]}
\makeatother


    % Exact colors from NB
    \definecolor{incolor}{rgb}{0.0, 0.0, 0.5}
    \definecolor{outcolor}{rgb}{0.545, 0.0, 0.0}



    
    % Prevent overflowing lines due to hard-to-break entities
    \sloppy 
    % Setup hyperref package
    \hypersetup{
      breaklinks=true,  % so long urls are correctly broken across lines
      colorlinks=true,
      urlcolor=urlcolor,
      linkcolor=linkcolor,
      citecolor=citecolor,
      }
    % Slightly bigger margins than the latex defaults
    
    \geometry{verbose,tmargin=1in,bmargin=1in,lmargin=1in,rmargin=1in}
    
    

    \begin{document}
    
    
    \maketitle
    
    

    
    \section{Sprawozdanie}\label{rob-klasyfikacja-bayesa-paweux142-paczuski-271028}

    \subsection{Eliminacja wartości
odstających}\label{eliminacja-wartoux15bci-odstajux105cych}

Ze zbioru uczącego usunięto próbki 642 (wartości skrajnie małe dla
większości cech) oraz 186 (wartości skrajnie duże dla większości cech).
Praktyczność usunięcia tych dwóch próbek widoczna jest skali wykresów
generowanych przez plot2features. Przed eliminacją dane tworzyły dwa
zgrupowania punktów: odstających i reszty. Po usunięciu wartości
odstających skala wykresów pozwala zauważyć, jak wyglądają różnice
między klasami wcześniej niewidoczne z racji swoje.

\subsection{Błędy dla kombinacji}\label{bux142ux119dy-dla-kombinacji}

Za pomocą nchoosek(liczba\_cech,2) wygenerowano wszystkie możliwe pary
cech i policzono dla nich błędy trzech podejść do generowania parametrów
dla klasyfikatora Bayesa:

\begin{longtable}[]{@{}llll@{}}
\toprule
kombinacja & pdfindep & pdfmulti & pdfparzen w=0.001\tabularnewline
\midrule
\endhead
2 3 & 0.17818 & 0.17873 & 0.21765\tabularnewline
2 4 & 0.02631 & 0.00493 & 0.02412\tabularnewline
2 5 & 0.07565 & 0.05975 & 0.27083\tabularnewline
2 6 & 0.13706 & 0.10197 & 0.27686\tabularnewline
2 7 & 0.10252 & 0.09758 & 0.27686\tabularnewline
2 8 & 0.22752 & 0.22862 & 0.27686\tabularnewline
3 4 & 0.02138 & 0.02083 & 0.01699\tabularnewline
3 5 & 0.13158 & 0.12774 & 0.23794\tabularnewline
3 6 & 0.15570 & 0.14912 & 0.31469\tabularnewline
3 7 & 0.16228 & 0.15406 & 0.31469\tabularnewline
3 8 & 0.28125 & 0.28399 & 0.31469\tabularnewline
4 5 & 0.15406 & 0.14583 & 0.23958\tabularnewline
4 6 & 0.15844 & 0.15132 & 0.26864\tabularnewline
4 7 & 0.23136 & 0.21217 & 0.26864\tabularnewline
4 8 & 0.21272 & 0.21272 & 0.26864\tabularnewline
5 6 & 0.37500 & 0.31305 & 0.47149\tabularnewline
5 7 & 0.32511 & 0.26919 & 0.47149\tabularnewline
5 8 & 0.38158 & 0.37829 & 0.47149\tabularnewline
6 7 & 0.32675 & 0.28947 & 0.54331\tabularnewline
6 8 & 0.40077 & 0.40241 & 0.57346\tabularnewline
7 8 & 0.36842 & 0.36678 & 0.54331\tabularnewline
\bottomrule
\end{longtable}

Najlepiej klasyfikowana jest zatem kobinacja cech 3 i 4.

\subsection{Klasyfikacja w oparciu o dwie wybrane
cechy}\label{klasyfikacja-w-oparciu-o-dwie-wybrane-cechy}

Wybrano cechy 3 i 4 i dla nich policzono błędy klasyfikatora Bayesa.
Dodatkowo, za pomocą funkcji toClient, dokonano projekcji wyników
klasyfikatora na cztery etykiety dostarczone przez klienta (oryginalnie
w danych było ich 8, co było spowodowane różnicą w sposobie otrzymania
danych), co skutkowało zmniejszeniem błędu klasyfikacji dlatego, że
błędne przestały być wyniki, którym wcześniej klasyfikator przypisywał
równoważne dla klienta klasy.

Dla prawdopodobieństwa \texttt{apriori\ =\ 0.25}

\begin{longtable}[]{@{}llll@{}}
\toprule
cechy & pdfindep & pdfmulti & pdfparzen w=0.001\tabularnewline
\midrule
\endhead
3 4 & 0.021382 & 0.020833 & 0.016996\tabularnewline
\bottomrule
\end{longtable}

używając etykiet klienta

\begin{longtable}[]{@{}llll@{}}
\toprule
cechy & pdfindep & pdfmulti & pdfparzen w=0.001\tabularnewline
\midrule
\endhead
3 4 & 0.010417 & 0.009868 & 0.006031\tabularnewline
\bottomrule
\end{longtable}

\subsection{Redukcja zbioru
trenującego}\label{redukcja-zbioru-trenujux105cego}

Zbadano wpływ redukcji zbioru trenującego na wyniki klasyfikacji. Tabele
zawierają uśrednione wyniki dla klasyfikacji z użyciem zredukowanego
zbioru trenującego dla pięciu prób.

\begin{longtable}[]{@{}lllll@{}}
\toprule
czesc & pdfindep & std & min & max\tabularnewline
\midrule
\endhead
0.1000 & 0.0257 & 0.0049 & 0.0203 & 0.0323\tabularnewline
0.2500 & 0.0205 & 0.0011 & 0.0192 & 0.0219\tabularnewline
0.5000 & 0.0213 & 0.0008 & 0.0203 & 0.0225\tabularnewline
\bottomrule
\end{longtable}

\begin{longtable}[]{@{}lllll@{}}
\toprule
czesc & pdfmulti & std & min & max\tabularnewline
\midrule
\endhead
0.1000 & 0.0258 & 0.0046 & 0.0214 & 0.0329\tabularnewline
0.2500 & 0.0202 & 0.0009 & 0.0192 & 0.0214\tabularnewline
0.5000 & 0.0197 & 0.0013 & 0.0181 & 0.0214\tabularnewline
\bottomrule
\end{longtable}

\begin{longtable}[]{@{}lllll@{}}
\toprule
czesc & pdfparzen w=0.001 & std & min & max\tabularnewline
\midrule
\endhead
0.1000 & 0.0402 & 0.0044 & 0.0351 & 0.0471\tabularnewline
0.2500 & 0.0283 & 0.0022 & 0.0252 & 0.0312\tabularnewline
0.5000 & 0.0203 & 0.0008 & 0.0197 & 0.0214\tabularnewline
\bottomrule
\end{longtable}

używając etykiet klienta

\begin{longtable}[]{@{}lllll@{}}
\toprule
czesc & pdfindep & std & min & max\tabularnewline
\midrule
\endhead
0.1000 & 0.0150 & 0.0057 & 0.0088 & 0.0241\tabularnewline
0.2500 & 0.0102 & 0.0010 & 0.0088 & 0.0115\tabularnewline
0.5000 & 0.0103 & 0.0010 & 0.0088 & 0.0110\tabularnewline
\bottomrule
\end{longtable}

\begin{longtable}[]{@{}lllll@{}}
\toprule
czesc & pdfmulti & std & min & max\tabularnewline
\midrule
\endhead
0.1000 & 0.0135 & 0.0053 & 0.0082 & 0.0225\tabularnewline
0.2500 & 0.0089 & 0.0012 & 0.0071 & 0.0099\tabularnewline
0.5000 & 0.0098 & 0.0011 & 0.0088 & 0.0115\tabularnewline
\bottomrule
\end{longtable}

\begin{longtable}[]{@{}lllll@{}}
\toprule
czesc & pdfparzen w=0.001 & std & min & max\tabularnewline
\midrule
\endhead
0.1000 & 0.0249 & 0.0020 & 0.0219 & 0.0214\tabularnewline
0.2500 & 0.0167 & 0.0024 & 0.0137 & 0.0214\tabularnewline
0.5000 & 0.0091 & 0.0012 & 0.0071 & 0.0214\tabularnewline
\bottomrule
\end{longtable}

Im większy zbior trenujący, tym mniejszy błąd klasyfikacji. Największa
wrażliwość na zwiększanie rozmiaru zbioru trenującego obserwowana jest
dla metody z użyciem okna Parzena -- wartość odchylenia standardowego
używanego do oszacowania pdf uzależniona jest bezpośrednio od liczby
próbek.

\subsection{Różne szerokości okna
Parzena}\label{ruxf3ux17cne-szerokoux15bci-okna-parzena}

\begin{longtable}[]{@{}ll@{}}
\toprule
parzen width & error\tabularnewline
\midrule
\endhead
0.000100 & 0.014254\tabularnewline
0.000500 & 0.016996\tabularnewline
0.001000 & 0.020285\tabularnewline
0.005000 & 0.044408\tabularnewline
0.010000 & 0.043311\tabularnewline
\bottomrule
\end{longtable}

używając etykiet klienta

\begin{longtable}[]{@{}ll@{}}
\toprule
parzen width & error\tabularnewline
\midrule
\endhead
0.000100 & 0.004386\tabularnewline
0.000500 & 0.005482\tabularnewline
0.001000 & 0.008772\tabularnewline
0.005000 & 0.031250\tabularnewline
0.010000 & 0.030154\tabularnewline
\bottomrule
\end{longtable}

Im mniejsza szerokość okna Parzena, tym mniejszy błąd klasyfikacji.

\subsection{Dwukrotnie większe prawdopodobieństwo apriori dla maści
czarnych}\label{dwukrotnie-wiux119ksze-prawdopodobieux144stwo-apriori-dla-maux15bci-czarnych}

Uśrednione dla pięciu prób wyniki klasyfikacji uzyskane przy użyciu
zredukownego zbioru testowego.

\begin{longtable}[]{@{}llll@{}}
\toprule
pdfindep & std & min & max\tabularnewline
\midrule
\endhead
0.0156 & 0.0034 & 0.0102 & 0.0183\tabularnewline
\bottomrule
\end{longtable}

\begin{longtable}[]{@{}llll@{}}
\toprule
pdfmulti & std & min & max\tabularnewline
\midrule
\endhead
0.0167 & 0.0024 & 0.0132 & 0.0190\tabularnewline
\bottomrule
\end{longtable}

\begin{longtable}[]{@{}llll@{}}
\toprule
pdfparzen w=0.001 & std & min & max\tabularnewline
\midrule
\endhead
0.0186 & 0.0032 & 0.0139 & 0.0227\tabularnewline
\bottomrule
\end{longtable}

etykiety klienta:

\begin{longtable}[]{@{}llll@{}}
\toprule
pdfindep & std & min & max\tabularnewline
\midrule
\endhead
0.0076 & 0.0020 & 0.0044 & 0.0095\tabularnewline
\bottomrule
\end{longtable}

\begin{longtable}[]{@{}llll@{}}
\toprule
pdfmulti & std & min & max\tabularnewline
\midrule
\endhead
0.0082 & 0.0015 & 0.0066 & 0.0095\tabularnewline
\bottomrule
\end{longtable}

\begin{longtable}[]{@{}llll@{}}
\toprule
pdfparzen w=0.001 & std & min & max\tabularnewline
\midrule
\endhead
0.0107 & 0.0010 & 0.0095 & 0.0117\tabularnewline
\bottomrule
\end{longtable}

Zmiana pradobodobieństwa (oraz redukcja części zbioru testowego) miała
pozytywny wpływ na wyniki klasyfikacji, co pozwala stwierdzić, że użyte
prawdopodobieństwo lepiej oddaje naturę klasyfikowanych danych.

\subsection{Normalizacja danych}\label{normalizacja-danych}

Odchylenie standardowe dla cech 3 i 4: \texttt{0.00092},
\texttt{0.00095}

Odchylenie standardowe w poszczególnych klasach cech 3 i 4

\begin{longtable}[]{@{}lll@{}}
\toprule
klasa & std cecha 3 & std cecha 4\tabularnewline
\midrule
\endhead
1 & 0.000063 & 0.000186\tabularnewline
2 & 0.000162 & 0.000003\tabularnewline
3 & 0.000022 & 0.000085\tabularnewline
4 & 0.000012 & 0.000098\tabularnewline
5 & 0.000023 & 0.000021\tabularnewline
6 & 0.000255 & 0.000002\tabularnewline
7 & 0.000010 & 0.000112\tabularnewline
8 & 0.000126 & 0.000009\tabularnewline
\bottomrule
\end{longtable}

błędy klasyfikacji dla znormalizowanych cech 3 i 4

\begin{longtable}[]{@{}llll@{}}
\toprule
cechy & pdfindep & pdfmulti & pdfparzen w=0.001\tabularnewline
\midrule
\endhead
3 4 & 0.021382 & 0.020833 & 0.020285\tabularnewline
\bottomrule
\end{longtable}

błędy klasyfikacji dla znormalizowanych cech 3 i 4 korzystając z etykiet
klienta

\begin{longtable}[]{@{}llll@{}}
\toprule
cechy & pdfindep & pdfmulti & pdfparzen w=0.001\tabularnewline
\midrule
\endhead
3 4 & 0.010417 & 0.009868 & 0.004386\tabularnewline
\bottomrule
\end{longtable}

Bardzo małe wartości odchylenia standardowego w klasach cech 3 i 4 sprawiają, że normalizacja danych nie jest specjalnie potrzebna, o czym świadczy brak różnicy w wynikach klasyfikacji dla przyjętej liczby cyfr znaczących.

\subsection{1NN vs Bayes}\label{nn-vs-bayes}

Błąd klasyfikatora 1NN wyniósł: \texttt{ercf\_1nn\ =\ 0.018092} Gdy
zastosujemy etykiety klienta: \texttt{ercf\_1nn\_client\ =\ 0.004385}

Klasyfikator uzyskje 1NN porównywalne wyniki do klasyfikatora Bayesa
zakładającego \texttt{apriori\ =\ 0.25}, jednak przy przejściu do
etykiet klienta widać zauważalną różnicę błędów na korzyść 1NN, co
sugeruje, że problematyczne dla 1NN punkty pochodzą z równoważnych dla
klienta klas.


    % Add a bibliography block to the postdoc
    
    
    
    \end{document}
